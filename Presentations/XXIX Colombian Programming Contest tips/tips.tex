\documentclass{beamer}
\usepackage[utf8]{inputenc}
\usepackage[spanish]{babel}
\usepackage{hyperref}
\usepackage{verbatim}
\usepackage{listings}

\setbeamercovered{invisible}
\usetheme{Frankfurt}
\usefonttheme{serif}

\newcommand{\source}[1]{
	\verbatiminput{#1}
	%\dotfill
}

% Configurar los listings (Códigos)
\renewcommand{\lstlistingname}{Code}
\lstset{
	language=C++,               % Lenguaje
	basicstyle=\ttfamily\tiny,  % Tipo de fuente
	keywordstyle=\color{blue},  % Color de palabras clave
	stringstyle=\color{red},    % Color de strings
	commentstyle=\color{gray},  % Color de comentarios
	showstringspaces=false,     % No muestrar el _ cuando el string tiene espacios
	breaklines = true,          % Partir las líneas largas
	breakatwhitespace=true,	    % Partir las líneas en un espacio
	numbers=left,				% Numerar las líneas a la izq
	numberstyle=\tiny,			% Poner los números de las líneas pequeños
	numberblanklines=true,      % Numerar las líneas en blanco
	columns=fullflexible,       % No perder el formato al dejar los espacios
	keepspaces=true,   			% Dejar los espacios insertados
	frame=tb,					% Poner el recuadro
}

\AtBeginSection[]{%
  \begin{frame}<beamer>
    \frametitle{Content}
    \tableofcontents[sectionstyle=show/hide,subsectionstyle=hide/show/hide]
  \end{frame}
  \addtocounter{framenumber}{-1}% If you don't want them to affect the slide number
}

\title{Programming Seminar}
\subtitle{XXIX ACIS Colombian Programming Contest tips}
\author{Santiago Vanegas Gil\\Esteban Foronda Sierra}

\institute{EAFIT University}
\date{Monday, August 31st 2015}

\begin{document}

\begin{frame}
	\titlepage
\end{frame}

\begin{frame}
	\frametitle{Content}
	\tableofcontents
\end{frame}

\begin{frame}[fragile]
	\frametitle{Set up}
	\begin{block}{Get ready}
		Open your favourite editor.\\
		\emph{We recommend Dev C++ or CodeBlocks for C/C++\\
			NetBeans or Eclipse for Java}
	\end{block}
\end{frame}

\begin{section}{Template}
\begin{subsection}{What is a template?}
\begin{frame}[fragile]
	\frametitle{What is a template?}
	\begin{block}{What is a template?}
		The template is the code contained by every program in the contest, i.e. a base code.\\
		For example:\\
		\emph{The main method should be included in all programs.}
		\begin{lstlisting}
int
main() {
  // Your code goes here.
  return 0;
}
		\end{lstlisting}
	\end{block}
\end{frame}
\end{subsection}

\begin{subsection}{Writing your template}
\begin{frame}[fragile]
	\frametitle{Let's write a template}
	Let's imagine that we already are in the contest. How would the editor be configured?\\
	\begin{block}{Test 1}
		\textbf{5 minutes}\\
		Each team should write a template for their favourite programming language. \emph{(We hope it is C++)}\\
		\emph{The first team writting a good template and configuring the editor will win.}
		\pause
	\end{block}
\end{frame}
\end{subsection}

\begin{subsection}{Recommended template}
\begin{frame}[fragile]
	\frametitle{A fast and enough template}
	\begin{itemize}
		\item Did you include lots of header files?
		\item Did you write \texttt{using namespace std;}?
		\item Did you write the main method?
	\end{itemize}
	\pause
	\begin{block}{Template}
		\source{./src/template.cpp}
	\end{block}
\end{frame}
\end{subsection}
\end{section}

\begin{section}{Input}
\begin{subsection}{Types of input}
\begin{frame}[fragile]
	\frametitle{Common types of input}
	In a programming contest, the most common input method is the standard one, i.e. reading from \textbf{standard input}. However, you could be required to \textbf{read from a file}, let's see how to do it.
\end{frame}
\end{subsection}

\begin{subsection}{Reading from a file}
\begin{frame}[fragile]
	\frametitle{Reading from a file in C++}
	You can use either input and output file streams (\texttt{ifstream}, \texttt{ofstream}) or just redirect the standard input and output to a file.
	\pause
	\begin{block}{Redirect standard streams}
		\source{./src/fileredirect.cpp}
	\end{block}
	\pause
	\begin{block}{Use new streams}
		\source{./src/filestreams.cpp}
	\end{block}
\end{frame}

\begin{frame}[fragile]
	\frametitle{Reading from a file in Java}
	You should use the \texttt{FileInputStream} class to read from a file.
	\pause
	\begin{block}{Redirect standard streams}
		\source{./src/fileinputstream.java}
	\end{block}
\end{frame}
\end{subsection}

\begin{subsection}{Reading from standard input}
\begin{frame}[fragile]
	\frametitle{Let's read some inputs}
	Use any reading method.
	\pause
	\begin{block}{Test 2}
		\textbf{4 minutes}\\
		Each team shoud write a program that reads from standard input an arbitrary set of integers and strings.\\
		You don't know how many elements are in the input, you only have to print each one in a single line.\\
		An element is a number or a string that is separated from another element by at least one space or line break.\\
	\end{block}
	\begin{tabular}{ l c c c c c c l }
		\textbf{Sample input} & & & & & & & \textbf{Sample output} \\
		\texttt{hi 251} & & & & & & & hi \\
		\texttt{mornin6a} & & & & & & & 251 \\
		\texttt{read 1 this} & & & & & & & mornin6a \\
		& & & & & & & read \\
		& & & & & & & 1 \\
		& & & & & & & this \\
	\end{tabular}
\end{frame}

\begin{frame}[fragile]
	\frametitle{Reading from standard input in C++}
	You can use \texttt{cin}, \texttt{scanf}, \texttt{getline}. Depending on your needs.\\
	Take care of:
	\begin{itemize}
		\item Usage of \texttt{cin} with \texttt{getline}: \emph{use cin.ignore()}
		\item Usage of \texttt{cin} with \texttt{scanf}: \emph{write} \texttt{ios::sync\_with\_stdio(false);}
			\emph{in the first line of your main method when using only} \texttt{cin}
			\emph{to get a better performance.}
	\end{itemize}
\end{frame}

%\begin{frame}[fragile]
%	\frametitle{Reading from a file in Java}
%	You should use the \texttt{FileInputStream} class to read from a file.
%	\pause
%	\begin{block}{Redirect standard streams}
%		\source{./src/fileinputstream.java}
%	\end{block}
%\end{frame}
\end{subsection}
\end{section}
%Not efficient method is to use the standard \texttt{Scanner}, but if you need a better performance you could use \texttt{BufferedReader}.
\end{document}