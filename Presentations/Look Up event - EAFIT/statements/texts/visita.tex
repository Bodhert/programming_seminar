\begin{problem}{Google en EAFIT}{entrada estándar}{salida estándar}{1 segundo}{google}

El pasado miércoles 20 de agosto de 2014 en la Universidad EAFIT, se tuvo una visita por parte de Google, donde se planteó un problema de programación, el cual vamos a llevar a cabo de nuevo.\\
\newline
Dada cierta cantidad de líneas, que contiene cada una al menos una palabra, se deben imprimir las líneas en el mismo orden de entrada y \emph{sin repetirlas}, es decir, si una línea aparece más de una vez, sólo debe ser mostrada en la primera ocurrencia.

\InputFile
La entrada está compuesta por varias líneas (máximo 100), una línea no puede tener un tamaño mayor a 100 caracteres. Los caracteres permitidos son únicamente mayúsculas y minúsculas del alfabeto inglés.\\
\textit{Se debe leer hasta fin de archivo (EOF).}

\OutputFile
La salida contiene la cantidad de líneas que deben ser impresas resolviendo el problema, por consiguiente, \emph{no deben haber líneas repetidas en la salida}.

\Example

\begin{example}
\exmp{
Se ve muy feo escribir sin tildes
Hola esto es una linea
Hola esto es otra linea
Hola esto es una linea
Un problema facil
Se ve muy feo escribir sin tildes
}{
Se ve muy feo escribir sin tildes
Hola esto es una linea
Hola esto es otra linea
Un problema facil
}%
\end{example}

\end{problem}