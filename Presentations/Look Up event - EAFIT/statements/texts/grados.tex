\begin{problem}{Teoría de los seis grados}{entrada estándar}{salida estándar}{3 segundos}{grados}

Existe una hipótesis llamada \textbf{"Seis grados de separación"} que intenta probar que cualquier persona en la tierra puede estar conectado con otra a través de una cadena de conocidos que no tiene más de cinco intermedios, es decir, conectando a ambas personas con sólo seis (6) enlaces.\\

Conociendo todas las relaciones que tienen las personas, hoy queremos determinar si, para dos personas, se cumple la teoría.\\

\InputFile

La primera línea de la entrada contiene un entero $n$ $(2 \leq n \leq 100)$, representando la cantidad de personas que existen en el caso de prueba.\\
Después, siguen $n$ líneas, cada una con el nombre de una persona. Se garantiza que la longitud del nombre no excede 20 caracteres, no se repiten nombres y estos están compuestos únicamente de letras mayúsculas o minúsculas del alfabeto inglés.\\
\newline
A continuación, sigue un entero $m$ $(1 \leq m \leq 500)$ que son la cantidad de relaciones que se van a describir. $m$ líneas siguen, cada una describiendo una relación, compuesta por dos nombres $a$ y $b$ separados por un espacio, esta relación significa que la persona $a$ conoce a la $b$ asi como $b$ conoce a $a$.\\
\newline
Finalmente, sigue un entero $q$ $(1 \leq q \leq 50)$, que indica la cantidad de consultas que se quieren hacer. Después de esto, seguirán $q$ líneas conteniendo una consulta cada una. Una consulta está compuesta por dos nombres separados por un espacio, se garantiza que los nombres son distintos y pertenecen a los dados por el caso de prueba.\\

\OutputFile

La salida debe contener $q$ líneas, cada una diciendo \texttt{"Aceptada"} si las personas involucradas en la consulta pueden conocerse en seis (6) o menos pasos, o \texttt{"Rechazada"} de lo contrario.

\Example

\begin{example}
\exmp{
10
Roxana
Esteban
Camila
Carlos
Felipe
Laura
Alicia
Santiago
Alejandra
Juliana
8
Roxana Esteban
Esteban Camila
Camila Carlos
Juliana Alejandra
Felipe Carlos
Felipe Laura
Alicia Santiago
Alicia Laura
4
Santiago Laura
Alicia Esteban
Roxana Santiago
Alejandra Felipe
}{
Aceptada
Aceptada
Rechazada
Rechazada
}%
\end{example}

\end{problem}