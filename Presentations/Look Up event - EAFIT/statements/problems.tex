\documentclass[12pt,letterpaper,oneside]{article}
\usepackage[english]{babel}
\usepackage[utf8]{inputenc}
\usepackage{olymp}
\usepackage[dvips]{graphicx}
\usepackage{color}
\usepackage{colortbl}
\usepackage{ulem}
\usepackage{amsmath}
%\usepackage{expdlist}
%\usepackage{mfpic}
%\usepackage{comment}

% Treat all pictures as MPS (metapost) when using PDFLaTeX tool
\ifx\pdftexversion\undefined
\else
 \DeclareGraphicsRule{*}{mps}{*}{}
\fi

\renewcommand{\contestname}{
LOOK UP - Maratón de Programación - EAFIT -  Agosto 30, 2014
}

% Copied from 'amstex.tex' \bmod declaration
\def\bdiv{\mskip-\medmuskip\mkern5mu\mathbin{\mathrm{div}}\mkern5mu\mskip-\medmuskip}
% End of copied text

\renewcommand{\t}{\texttt}

\begin{document}


\thispagestyle{empty}
\begin{center}
	\LARGE\textbf{\center Maratón de programación LOOK UP - EAFIT}
	\LARGE\textbf{\center Agosto 30, 2014}
	\vspace*{\fill}
	\LARGE\textbf{\center Comentarios generales}
	\begin{itemize}
		\item Este conjunto contiene 4 problemas
		\item La maratón tiene una duración de 150 minutos en total.
               	\item La tabla de resultados será congelada después de 120 minutos.
                	\item Los envios no serán respondidos si fueron enviados después del pasados 130 minutos de competencia. (Serán respondidos cuando la competencia acabe).
		\item Si están programando en Java, recuerden que el nombre del código fuente como el nombre de la clase que contiene el método $main$ debe ser nombrado igual a como se describió en cada problema.
		\item Buena suerte para todos y diviértanse.
	\end{itemize}
	\quad \\ \quad \\
	\vspace*{\fill}
\end{center}
\newpage

\thispagestyle{empty}
\vspace*{\fill}
\begin{center}
	\LARGE{Página dejada intencionalmente en blanco.}
\end{center}
\vspace*{\fill}
\newpage

\input texts/stack.tex
\input texts/grados.tex
\input texts/visita.tex
\input texts/paseo.tex

\end{document}
