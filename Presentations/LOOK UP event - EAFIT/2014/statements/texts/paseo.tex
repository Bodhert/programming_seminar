\begin{problem}{El paseo de Roxana}{entrada estándar}{salida estándar}{3 segundos}{paseo}

Roxana está planeando un paseo a su ciudad favorita, ella conoce tanto la ciudad, tanto asi que conoce las rutas que llegan a ella. Lastimosamente, ella posee un presupuesto muy apretado, por esto, te pidió el favor de calcular cuál sería el mínimo costo para realizar dicho viaje.

\InputFile

La primera línea de la entrada contiene dos enteros $n$ $(1 \leq n \leq 100)$ y $m$ $(1 \leq m \leq 500)$, representando el número de ciudades y la cantidad de rutas que conoce respectivamente. Las ciudades están numeradas de 1 hasta $n$.\\
Después, siguen $m$ líneas, cada una contiene tres enteros $a$, $b$ y $w$ $(1 \leq a, b \leq n)$, $(w \geq 0)$ que describen respectivamente, la ciudad origen, la ciudad destino y el costo de la ruta. Lastimosamente, las rutas no son en ambos sentidos.

\OutputFile

Deberás imprimir el mínimo costo de ir de la ciudad 1 hasta la ciudad $n$, si no es posible hacer este viaje el programa debe imprimir -1.

\Example

\begin{example}
\exmp{
4 5
1 2 3
1 3 2
2 3 4
2 4 2
3 4 1
}{
3
}%
\end{example}

\Example

\begin{example}
\exmp{
4 5
1 2 4
1 3 1
2 3 5
4 2 6
4 3 7
}{
-1
}%
\end{example}
\end{problem}