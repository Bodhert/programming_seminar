\begin{problem}{Roxana y las pilas}{entrada estándar}{salida estándar}{3 segundos}{pilas}

Roxana es una pequeña niña que está aprendiendo estructuras de datos. Ella te pidió que la ayudaras con un programa que simula cómo funciona una pila.

Roxana te dirá $n$ instrucciones, cada una puede ser de sólo un tipo. Veamos qué tipo de instrucciones conoce ella:
\begin{itemize}
	\item Cuando Roxana dice \textbf{PUSH}, este será seguido por un espacio y un número $k$ $(1 \leq k \leq 1000)$; el cual debe ser insertado en la pila. \emph{En este tipo de instrucción no tienes que imprimir nada}.
	\item Cuando Roxana dice \textbf{POP}, debes remover el último elemento de la pila, \emph{Pero no imprimirlo}. Si no hay elementos en la pila debes ignorar esta instrucción.
	\item Cuando Roxana dice \textbf{TOP}, \emph{debes imprimir, en una línea, el último elemento de la pila sin removerlo}. Si la pila está vacía, debes imprimir \texttt{"EMPTY"} sin comillas.
\end{itemize}

\InputFile

La primera línea de la entrada contiene un entero $n$ $(1 \leq n \leq 100)$, representando el número de instrucciones que Roxana te dirá.\\
Después, siguen $n$ líneas, cada una contiene cualquiera de las instrucciones explicadas anteriormente.

\OutputFile

La salida debe contener el resultado obtenido ejecutando las instrucciones.

\Example

\begin{example}
\exmp{
6
PUSH 1
TOP
POP
TOP
PUSH 5
TOP
}{
1
EMPTY
5
}%
\end{example}
\end{problem}