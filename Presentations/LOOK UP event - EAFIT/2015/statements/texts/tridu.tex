\begin{problem}{Tri-du}{standard input}{standard output}{tridu}

Tri-du is a card game inspired in the popular game of Truco. The game uses a normal deck of 52
cards, with 13 cards of each suit, but suits are ignored. What is used is the value of the cards,
considered as integers between 1 to 13.\\
In the game, each player gets three cards. The rules are simple:

\begin{itemize}
  \item A Three of a Kind (three cards of the same value) wins over a Pair (two cards of the same
        value).
  \item A Three of a Kind formed by cards of a larger value wins over a Three of a Kind formed by
        cards of a smaller value.
  \item A Pair formed by cards of a larger value wins over a Pair formed by cards of a smaller
        value.
\end{itemize}

Note that the game may not have a winner in many situations; in those cases, the cards are returned
to the deck, which is re-shuffled and a new game starts.\\
A player received already two of the three cards, and knows their values. Your task is to write a
program to determine the value of the third card that maximizes the probability of that player
winning the game.

\InputFile

The input contains several test cases. In each test case, the input consists of a single line, which
contains two integers $A\ (1 \leq A \leq 13)$ and $B\ (1 \leq B \leq 13)$ that indicates the value of
the two first received cards.

\OutputFile

For each test case in the input, your program must produce a single line, containing exactly one
integer, representing the value of the card that maximizes the probability of the player winning the
game.

\Example

\begin{example}
\exmp{
10 7
2 2
}{
10
2
}%
\end{example}

\end{problem}
