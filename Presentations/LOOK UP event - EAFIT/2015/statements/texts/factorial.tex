\begin{problem}{Factorial}{standard input}{standard output}{factorial}

The $factorial$ of a positive integer number $N$, denoted as $N!$, is defined as the product of all
positive integer numbers smaller or equal to $N$. For example $4! = 4 \times 3 \times 2 \times 1 =
24$.\\
Given a positive integer number $N$, you have to write a program to determine the smallest number
$k$ so that $N = a_1! + a_2! + .\ .\ .\ + a_k!$, where every $a_i$, for $1 \leq i \leq k$, is a
positive integer number.

\InputFile

The input consists of several test cases. A test case is composed of a single line, containing one
integer number $N\ (1 \leq N \leq 105)$.\\

\OutputFile

For each test case in the output your program must output the smallest quantity of factorial numbers
whose sum is equal to $N$.\\

\Example

\begin{example}
\exmp{
10
25
}{
3
2
}%
\end{example}

\end{problem}
